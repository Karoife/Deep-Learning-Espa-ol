\chapter{Introducción}
\label{chp:introduccion}
Los inventores ha soñado en crear una maquina que piensa. Este deseo se remonta al menos a la antigua Grecia. Las figuras míticas de Pigmalión, Dédalo y Hefesto tal vez son interpretados como inventores legendarios y Galatea, Talon y Pandora podrían ser consideras como vida artificial (\cite{2005metamorphoses}, \cite{sparkes2013red}, \cite{1997works} ).\\

Cuando se concibieron por primera vez las computadoras programables, las personas se preguntaban si estas podrían volverse inteligentes, más de cien años antes que se construyera una \cite{menabrea1843sketch}. Hoy la \textit{inteligencia artificial} (IA) es un campo prospero con muchas aplicaciones practicas y temas de investigación activos. Nosotros buscamos un software inteligente que automatice las labores rutinarias, entienda discursos o imágenes, haga diagnósticos en medicina y ayude en investigaciones científicas básicas. \\

En los primeros días de la inteligencia artificial, el campo abordó y resolvió problemas intelectualmente difíciles para el ser humano pero relativamente sencillos para una computadora - problemas que podían ser descritos por una lista de formulas matemáticas formales. EL verdadero desafió para la inteligencia artificial es probar poder resolver tareas que son sencillas de desempañar para las personas pero difíciles de describir formalmente- problemas que se resuelven intuitivamente, que se sienta automático, como reconocer palabras habladas o caras en imágenes.\\

Este libro se enfoca en la solución de ese tipo de problemas más intuitivos. Esta solución deja a las computadoreas aprender de la experiencia y comprender el mundo en terminos de la jerarquía de conceptos, con cada concepto definido en terminos de su relacion con conceptos más simples. Al recopilar conocimientos a partir de la experiencia, este enfoque evita la necesidad de que los operadores humanos especifiquen formalmente todo el conocimientos formal que la computadora necesita. La jerarquía de conceptos permite a la computadora aprender conceptos complicados construyendo estos a base de conceptos más simples. Si nosotros dibujamos un grafico mopstrando como estos conceptos son contruidos uno encima de otros, el grafico sería muy profundo, con muchas capas. Por esta razón, nosotros llamaso a esta aproximación a la IA \textit{aprendizaje profundo}.\\

Muchos de los sucesos dempranos de las IA toman lugar en ambientes relativamente esteriles y formales y no requieren computadoras para tener el conocimiento sobre el mundo. 